
%\documentclass[12pt]{beamer}
\documentclass[11pt]{article}
\usepackage[utf8]{inputenc}
\usepackage{graphicx}

\usepackage{amsmath}
\title{Jakob Hohwy: The Predictive Mind}
\author{\small Conrad Friedrich // Universität zu Köln \\ \small Seminar: Theorien der Wahrnehmung und Predictive Coding WS16/17 \\ \small Prof. Albert Newen // \small Ruhr-Universität Bochum}
\setlength\parindent{0pt}

\begin{document}

  \maketitle

\section{Perception and Bayesian Inference}


Zentrale Idee: Das Gehirn lässt sich als
    Hypothesen-Prüf-Mechanismus betrachten, der durchgehend damit
    beschäftigt ist, die Abweichung seiner Vorhersagen/Erwartungen
    (predictions) von seinen Sinneseindrücken zu \emph{minimieren}.


\subsection{Bayesian Inference}

  Zutaten für die Bayesianistische Inferenz:

  \begin{description}
  \item[$h_1,...,h_n$] Möglichen Hypothesen als Ursache eines
    Sinneseindrucks
  \item[$e$] Gegebener Sinneseindruck
  \item[$P(h_i)$] \emph{Prior:} Wahrscheinlichkeit einer Hypothese
    $h_i$, unabhängig davon, ob es die Ursache ist. Abhängig vom
    Hintergrundwissen.
  \item[$P(e|h_i)$] \emph{Likelihood:} Wahrscheinlichkeit, dass die in
    der Hypothese beschrieben Ursache so einen Sinneseindruck
    hervorrufen würde. Abhängig vom Hintergrundwissen. Maß dafür, wie
    gut eine Hypothese den Sinneseindruck \emph{vorhersagt}.
  \end{description}


  Vereinfachtes Bayes Theorem 

  \begin{align*}
    P(h_i|e) &= P(e|h_i)P(h_i) \\
    \mathrm{Posterior \: Probability} &= \mathrm{Likelihood} \times  \mathrm{Prior \: Probability}
  \end{align*}

  \begin{itemize}
  \item Hypothese $h_i$, für die $P(h_i|e)$ maximal ist, stellt
    plausibelste Ursache des Sinneseindrucks $e$ dar, gegeben das
    Hintergrundwissen des Agenten.
  \end{itemize}

\section{Objections}

\subsection{Einwand vom Intellektualistismus}

  \noindent Argument:

  \begin{enumerate}
  \item Bayesian Inference ist schwer und kognitiv anspruchsvoll.
  \item Wahrnehmung ist einfach und automatisch.
  \item Also: Bayesian Inference und Wahrnehmung sind verschieden.
  \end{enumerate}


\noindent  Hohwys Strategie:
  \begin{itemize}
  \item Fallstudien, deren \emph{beste Erklärung} eine automatische
    Inferenzleistung des Gehirns ist.
  \item Mindestens eine der beiden Prämissen ist dann nicht haltbar.
  \item Studien basieren auf dem Phänomen \emph{Binocular Rivalry}.
  \item Hohwy: Phänomen lässt sich am besten mit Bayesian Inference
    erklären.
  \end{itemize}

\noindent  Bayesianistisches Modell erklärt, warum ein konstantes Bild gesehen
  wird.

\subsection{Einwand vom Anthromorphismus}

\subsection{Einwand von der Phänomenologie}


\subsection{Einwand vom Skeptizismus}

\section{Prediction Error Minimization}


\end{document}
