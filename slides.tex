
%\documentclass[12pt]{beamer}
\documentclass[11pt, handout]{beamer}
\usepackage[utf8]{inputenc}
\usetheme{Copenhagen} 
\usecolortheme{dove}
\usepackage{graphicx}
\usepackage{wrapfig}
\usefonttheme[onlymath]{serif}% \setbeamercovered{transparent}


\newcommand{\prule}{\vspace{-8pt}\rule{50pt}{0.5pt}}



\AtBeginSection[]{
  \begin{frame}
  \vfill
  \centering
  \begin{beamercolorbox}[sep=8pt,center,shadow=true,rounded=true]{title}
    \usebeamerfont{title}\insertsectionhead\par%
  \end{beamercolorbox}
  \vfill
  \end{frame}
}

\title[Hohwy: The Predictive Mind]{Jakob Hohwy: The Predictive Mind}
\subtitle{Chapters 1\&2}
\author{Conrad Friedrich}
\institute[Uni Köln]{Universität zu Köln}


\begin{document}

\begin{frame}[plain]
  \maketitle
\end{frame}

\section{Perception and Bayesian Inference}

\subsection{Motivation}

\begin{frame}
  \begin{itemize}[<+->]
  \item Vereinheitlichende Theorie des Geistes: Wahrnehmung, Handlung,
    und ``alles mentale dazwischen'' (Auch Bewusstsein?).
  \item Zentrale Idee: Das Gehirn lässt sich als
    Hypothesen-Prüf-Mechanismus betrachten, der durchgehend damit
    beschäftigt ist, die Abweichung seiner Vorhersagen/Erwartungen
    (predictions) von seinen Sinneseindrücken zu \emph{minimieren}.
  \item Hier: Fokus auf Wahrnehmung.
  \item Die Sinneseindrücke formen Wahrnehmung nicht direkt, sondern
    sind Feedback zu den Erwartungen und Anfragen des Geistes ``an die
    Welt''.
  \end{itemize}
\end{frame}

\begin{frame}
  \begin{itemize}[<+->]
  \item Wahrnehmung besteht in (lässt sich am besten beschreiben als)
    unbewusster Inferenz auf die wahrscheinlichste \emph{Ursache}
    meiner \emph{rohen Sinneseindrücke} (und deren Vorhersage).
  \item Direkter, unmittelbarer Zugang nur zu den Sinneseindrücken,
    nicht zu den Dingen ``in der Welt''.
  \item \emph{Gewusst} (in einem starken Sinn, Gewissheit) werden nur
    die Effekte, d.h. Sinneseindrücke. Um etwas über die
    ``versteckten'' Ursachen zu erfahren, ist Inferenz nötig.
  \item Inferenz weniger stark als Gewissheit und inbesondere
    nicht-monoton.
  \item Denn: Zwischen Ursachen und Effekten besteht keine $1$:$1$
    Relation (sondern $n$:$m$).
  \item D.h. Verschiedene Ursachen können denselben Effekt haben, und
    eine Ursache verschiedene Effekte.
  \end{itemize}
\end{frame}

\begin{frame}[plain]
  \begin{center}
    \includegraphics[height=\paperheight]{bicycledistorted.jpeg}
  \end{center}
\end{frame}

\begin{frame}
  \begin{itemize}[<+->]
  \item Verschiedene Objekte, Zustände könnten diesen Sinneseindruck
    verursachen:
    \begin{itemize}[<+->]
    \item Ein Fahrrad, das im Gebüsch liegt
    \item Einzelne Fahrradteile, die irgendwie im Gebüsch hängen
      geblieben sind
    \item Ein ungewöhnlich genau koordinierter Schwarm Bienen
    \end{itemize}
  \item Wie kommen wir vom Sinneseindruck zum (offensichtlichen)
    Ergebnis, dass hier ein Fahrrad im Gebüsch liegt?
  \end{itemize}

\end{frame}


\begin{frame}
  \begin{itemize}[<+->]
  \item Also: Wir brauchen eine \emph{Inferenz} auf die `beste' Ursache.
  \item Not any old inference will do: Besondere zusätzliche
    Beschränkungen auf die Art der Inferenz.
    \begin{itemize}[<+->]
    \item Hintergrundwissen des Agenten.
    \item Möglichkeit, gute/richtige Inferenz von schlechter/falscher
      abzugrenzen.
    \end{itemize}
  \end{itemize}
\end{frame}


\begin{frame}
  \begin{itemize}[<+->]
  \item Gegeben Sinneseindruck und Hintergrundwissen soll auf die
    `richtige' Ursache des Sinneseindrucks geschlossen werden.
  \item Ranking der möglichen Ursachen nach Wahrscheinlichkeit.
  \item Die `beste' Ursache scheint guter Kandidat für den
    Wahrnehmungsinhalt zu sein.
  \end{itemize}

\end{frame}

\subsection{Bayesian Inference}

\begin{frame} 
  Zutaten für die Bayesianistische Inferenz:

  \begin{description}[xxxxxxxxx]
  \item[$h_1,...,h_n$] Möglichen Hypothesen als Ursache eines
    Sinneseindrucks
  \item[$e$] Gegebener Sinneseindruck
  \item[$P(h_i)$] \emph{Prior:} Wahrscheinlichkeit einer Hypothese
    $h_i$, unabhängig davon, ob es die Ursache ist. Abhängig vom
    Hintergrundwissen.
  \item[$P(e|h_i)$] \emph{Likelihood:} Wahrscheinlichkeit, dass die in
    der Hypothese beschrieben Ursache so einen Sinneseindruck
    hervorrufen würde. Abhängig vom Hintergrundwissen. Maß dafür, wie
    gut eine Hypothese den Sinneseindruck \emph{vorhersagt}.
  \end{description}

  \begin{itemize}[<+->]
  \item Daraus lässt sich für jedes $h_i$ die bedingte
    Wahrscheinlichkeit gegeben $e$ errechnen.
  \end{itemize}

\end{frame}

\begin{frame}

  Vereinfaches Bayes Theorem \\{\scriptsize (Standardvariante folgt
    direkt aus der Definition von bedingter Wahrscheinlichkeit)}

  \begin{align*}
    P(h_i|e) &= P(e|h_i)P(h_i) \\
    \mathrm{Posterior \: Probability} &= \mathrm{Likelihood} \times  \mathrm{Prior \: Probability}
  \end{align*}

  \begin{itemize}[<+->]
  \item Hypothese $h_i$, für die $P(h_i|e)$ maximal ist, stellt
    plausibelste Ursache des Sinneseindrucks $e$ dar, gegeben das
    Hintergrundwissen des Agenten.
  \end{itemize}

\end{frame}

\begin{frame}

\small

      \begin{itemize}[<+->]
      \item Bayes Theorem eigentlich
        \begin{equation*}
        P(h_i|e) = \frac{P(e|h_i)P(h_i)}{P(e)}
      \end{equation*}
      
      \item Wieso wird der Nenner $P(e)$  weggelassen?
      \item Für das reine Ranking der Hypothesen irrelevant, da konstanter Faktor.
      \item Mathematisch: Normalisierungskonstante, so dass $\sum_{i} P(h_i|e) = 1$ (und nicht weniger).
      \end{itemize}

\end{frame}

\begin{frame}
  Beispiel.
  \begin{description}[xxxxxxx]
  \item[$e$] {\small Sinneseindruck: Ein seltsames Klopfen.}
  \item[$h_1$] {\small Ein Specht klopft an der Wand.}
  \item[$h_2$] {\small Ein Einbrecher werkelt an der Tür.} \\ $...$
  \item[$h_{815}$] {\small Ich bin ein BIV und bekomme gerade einen
      elektrischen Stimulus entsprechend dem Sinneseindruck.}
  \end{description}

  {\small Angenommen, ich habe zuletzt viele Spechte in der
    Nachbarschaft bemerkt, aber eher wenige Einbrecher. Ich habe mal
    ein Seminar zum Skeptizismus besucht und halte das ganze für
    großen Humbug. Dann gilt für die \emph{Prior Probabilities}}

  \begin{center}
    $P(h_1) > P(h_2) \gg P(h_{815})$.
  \end{center}
\end{frame}

\begin{frame}
  Aber:
  \begin{center}
    $P(e|h_{815}) > P(e|h_1) \approx P(e|h_2)$.
  \end{center}
  \begin{itemize}[<+->]
  \item Da: Gegeben, ich bin \emph{tatsächlich} ein BIV, ist die
    Wahrscheinlichkeit für jeden Sinneseindruck, den ich habe, sehr
    hoch, sogar fast sicher.
  \item Also höher als die anderen bedingten Wahrscheinlichkeiten.
  \item Resultat des Rankings:
    \begin{center}
      $P(h_1|e) > P(h_2|e) > P(h_{815})$.
    \end{center}
  \item Die Spechthypothese erscheint am wahrscheinlichsten. Resultat
    der Baysianistischen Inferenz - unbewusste Wahrnehmungsinferenz.
  \end{itemize}
\end{frame}


\begin{frame}
  Was hat das mit Prediction (Vorhersage) zu tun? Immerhin heißt es
  \emph{Predictive} Coding/Processing/Mind.  In der
  Wahrscheinlichkeitstheorie:
  \begin{itemize}[<+->]
  \item \textbf{Inference}: Reasoning from effects to causes
    (\emph{latent} or \emph{hidden}).
  \item \textbf{Prediction}: Reasoning from causes to (future)
    effects.
  \item Schlussfolgern wie vorgestellt Beispiel für Inferenz. Beide
    Vorgänge für Modell der Wahrnehmung relevant (s. später:
    prediction error minimization)
  \end{itemize}
\end{frame}



\section{Objections}

\begin{frame}
  Beispiel für simple Probalistische Inferenz vom Effekt zur
  Ursache. Aber beschreibt das auch \emph{Wahrnehmung} adäquat? Hohwy
  nennt vier Einwände zu dieser Interpretation.  \footnotesize
  \begin{description}[<+->]
  \item[Einwand 1] Probalistische Inferenz scheint viel zu
    intellektualistisch zu sein, um einen automatischen Vorgang wie
    Wahrnehmung erklären zu können.
  \item[Einwand 2] Unangebrachte Antromorphisierung von Prozessen im
    Gehirn: Wieso sollte das Gehirn etwas glauben oder Schlussfolgern,
    dass wir nicht bewusst machen?
  \item[Einwand 3] Keine offensichtliche Erklärung der Phänomenologie
    von Wahrnehmung. Wahrnehmung fühlt sich wie etwas an, aber
    beschreibt die Inferenz nicht nur reine begriffliche Kategosierung
    von Input?
  \item[Einwand 4] Arbitrarität der \emph{Prior Beliefs}: Wie zeichnet
    dieses Modell probabilistisch konsistente (d.h. keine Verletzung
    der Kolmogorov-Axiome), aber völlig realitätsferne
    Wahrscheinlichkeitsfuntionen als schlecht/falsch aus?
  \end{description}
\end{frame}

\subsection{Einwand vom Intellektualistismus}

\begin{frame}
{\large 1. Intellectualist Objection}
  Zu intellektualistisch? Immediate Intuition von formaler
  Erkenntnistheorie, in der Degress of Belief Bayesianistisch Updatet
  werden. Hier aber perception: Inhalte mussen keinen propositionalen
  Gehalt haben, nicht bewusst zugaenglich sein, usw.  Ähnlichkeit zu
  Konditionalisierungregel in formaler Erkenntnistheorie. Unterschied:
  Keine Anforderung, dass $e$ oder $h$ Überzeugungen sind,
  möglicherweise nicht-propotisitional.
\end{frame}

\begin{frame}
  Argument:

  \begin{enumerate}[<+->]
  \item Bayesian Inference ist schwer und kognitiv anspruchsvoll.
    \begin{itemize}[<+->]
    \item Evidenz für notorisch schlechte Bayesian Inference bei
      Erwachsenen (Linda the Bank-Teller)
    \end{itemize}
  \item Wahrnehmung ist einfach und automatisch.
    \begin{itemize}[<+->]
    \item Kinder und auch Tiere können reliabel wahrnehmen. Außerdem
    
    \end{itemize}
  \item Also: Bayesian Inference und Wahrnehmung sind verschieden.
  \end{enumerate}

\end{frame}

\begin{frame}
  Hohwys Strategie:
  \begin{itemize}[<+->]
  \item Fallstudien, deren \emph{beste Erklärung} eine automatische
    Inferenzleistung des Gehirns ist.
  \item Mindestens eine der beiden Prämissen ist dann nicht haltbar.
  \item Studien basieren auf dem Phänomen \emph{Binocular Rivalry}.
  \item Hohwy: Phänomen lässt sich am besten mit Bayesian Inference
    erklären.
  \end{itemize}
\end{frame}

\begin{frame}
  Binocular Rivalry
  \begin{quote}
    If a picture of a house is shown to one eye and a picture of a
    face is shown to the other, then one should surely just see a
    face-house. But this is not what happens, as Porta and Wheatstone
    and many others have described. The brain somehow seems to decide
    that there are two distinct things out there, a face and a
    house—and perception duly alternates between seeing one or the
    other every few seconds, sometimes with periods of patchy rivalry
    in between.
  \end{quote}
  \begin{itemize}[<+->]
  \item Varianten von Diaz-Caneja (1928), Logothetis (1996), Denison,
    Piazza et al. (2011), Zhou, Jiang et al. (2010).
  \end{itemize}
   
\end{frame}

\begin{frame}

  Input linkes Auge / Rechtes Auge

  \begin{center}
    \includegraphics[height=0.2\paperheight]{facehouse.png}
  \end{center}
  `Intuitiv' erwartetes Perzept

  \begin{center}
    \includegraphics[height=0.2\paperheight]{facehousemash.png}
  \end{center}

  Stattdessen stabile sequentielle Wahrnehmung:
  \begin{center}
    \includegraphics[height=0.2\paperheight]{facehousedistinct.png}
  \end{center}
\end{frame}



\begin{frame}
  \small Hohwy: Für dieses Phänomen lässt sich eine plausible
  Bayesianistische Geschichte erzählen.

  \begin{description}[<+->]
  \item[Input]
    \begin{description}[<+->]
    \item[$e$] Gesicht/Haus linkes Auge/rechtes Auge

    \end{description}
  \item[Hypothesen]
    \begin{description}
    \item[$h_{f+h}$] Es ist ein Gesichts-Haus-Mix.
    \item[$h_f$] Es ist ein Gesicht.
    \item[$h_h$] Es ist ein Haus.
    \end{description}
  \item[Likelihoods] $P(e|h_f) \approx P(e|h_h) < P(e|h_{f+h})$.
  \item[Priors] $P(h_f) > P(h_h) \gg P(h_{f+h})$.

  \end{description}

  \begin{description}[<+->]
  \item[Posteriors] $P(h_f|e) > P(h_h|e) > P(h_{f+h}|e)$.
  \end{description}

\end{frame}

\begin{frame}
  Bayesianistisches Modell erklärt, warum ein konstantes Bild gesehen wird.

  \begin{itemize}[<+->]
  \item Alternierende Wahrnehmung wird in dieser simplen Version \emph{nicht} erklärt.
  \item Trotzdem sieht Hohwy das Erklärungspotentiel als starke Evidenz für probabilistisches Wahrnehmungsschließen.
  \item Weitere Evidenz: Probanden Priming-Effekt unterzogen, so dass sie mehrheitlich ein bestimmtes Bild sehen (Zhou, Jiang et al., 2010)
    \begin{itemize}[<+->]
    \item Input Bild Text Marker/Rose. Priming durch Rosen\emph{geruch}. Effekt durch Modell vorhergesagt.
    \end{itemize}
  \end{itemize}
\end{frame}



\subsection{Einwand vom Anthromorphismus}

\begin{frame}
{\large 2. Anthromorphist Objection}
  \begin{itemize}[<+->]
  \item Ist das Gehirn tatsächlich mit Probabilistischem Schließen
    beschäftigt, oder ist das unzulässige `Anthromorphisierung' des
    Gehirns?
  \item Howhy argumentiert (a) funktionalistisch und (b) strukturell.
  \item `Bayesian coding hypothesis': The brain represents sensory
    information probabilistically, in the form of probability
    distributions. \footnote{Knill/Pourget (2004)}
  \item Fragestellung in Computational Cognitive Neuroscience.
  \end{itemize}
\end{frame}

\subsection{Einwand von der Phänomenologie}

\begin{frame}
  {\large 3. Phenomenologist Objection}
  \begin{itemize}[<+->]
  \item Selbst wenn die Kategorisierung von Wahrnehmungsinhalten Bayesianistisch funktioniert, heißt das nicht, dass das Wahrnehmungserlebnis erklärbar wird. 
  \item ``It is not just that we see a car but that we see it, as a car, from our own perspective.'' (p. 26)
  \end{itemize}

\end{frame}

\subsection{Einwand vom Skeptizismus}

\begin{frame} {\large 4. Skepticist Objection}
  \begin{itemize}[<+->]
  \item Das bisher beschriebene Framework ist anfällig für
    \emph{pathological Priors}:
  \item Völlig absurde, realitätsfremde Priors zeichnen bestimmte
    Hypothesen als wahrscheinlich aus, gegeben die Sinneseindrücke,
    und können so bestärkt werden.
  \item Bootstrapping-Problem (ähnlich epistemischem Internalismus).
  \item Hohwys Antwort: Bisher nur allgemeines Bayesianisches
    Framework, \emph{Predictive Coding} erfordert mehr: `Reality
    Check'
  \item \emph{Prediction Error Minimization}.
  \end{itemize}
\end{frame}

\begin{frame}

  {\Large References}

  \begin{itemize}
  \item Tong, Nakayama et al. Binocular Rivalry and Visual Awareness
    in Human Extrastriate Cortex. Neuron (21), 1998.
  \item Knill and Pourget. The Bayesian brain: the role of uncertainty
    in neural coding and computation. TRENDS in Neurosciences (27),
    2004.
  \item Hohwy. The Predictive Mind. Oxford University Press, 2013.

  \end{itemize}


\end{frame}

\end{document}
