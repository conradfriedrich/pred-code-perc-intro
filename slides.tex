
%\documentclass[12pt]{beamer}
\documentclass[12pt, handout]{beamer}
\usepackage[utf8]{inputenc}
\usetheme{Copenhagen} 
\usecolortheme{dove}
\usepackage{graphicx}
\usefonttheme[onlymath]{serif}% \setbeamercovered{transparent}


\newcommand{\prule}{\vspace{-8pt}\rule{50pt}{0.5pt}}



\AtBeginSection[]{
  \begin{frame}
  \vfill
  \centering
  \begin{beamercolorbox}[sep=8pt,center,shadow=true,rounded=true]{title}
    \usebeamerfont{title}\insertsectionhead\par%
  \end{beamercolorbox}
  \vfill
  \end{frame}
}

\title[The Predictive Mind]{Jakob Hohwy: The Predictive Mind}
\subtitle{Chapters 1\&2}
\author{Conrad Friedrich}
\institute[Uni Köln]{Universität zu Köln}


\begin{document}

\begin{frame}[plain]
  \maketitle
\end{frame}

\section{Ein Problem der Wahrnehmung}

\begin{frame}
  \begin{itemize}[<+->]
  \item Vereinheitlichende Theorie des Geistes: Wahrnehmung, Handlung,
    und ``alles mentale dazwischen'' (Auch Bewusstsein?).
  \item Zentrale Idee: Das Gehirn lässt sich als
    Hypothesen-Prüf-Mechanismus betrachten, der durchgehend damit
    beschäftigt ist, die Abweichung seiner Vorhersagen/Erwartungen
    (predictions) von seinen Sinneseindrücken zu \emph{minimieren}.
  \item Hier: Fokus auf Wahrnehmung.
  \item Die Sinneseindrücke formen Wahrnehmung nicht direkt, sondern
    sind Feedback zu den Erwartungen und Anfragen des Geistes ``an die
    Welt''.
  \end{itemize}
\end{frame}

\begin{frame}
  \begin{itemize}[<+->]
  \item Wahrnehmung besteht in (lässt sich am besten beschreiben als)
    unbewusster Inferenz auf die wahrscheinlichste \emph{Ursache}
    meiner \emph{rohen Sinneseindrücke}.
  \item Direkter, unmittelbarer Zugang nur zu den Sinneseindrücken,
    nicht zu den Dingen ``in der Welt''.
  \item \emph{Gewusst} (in einem starken Sinn, Gewissheit) werden nur
    die Effekte, d.h. Sinneseindrücke. Um etwas über die
    ``versteckten'' Ursachen zu erfahren, ist Inferenz nötig.
  \item Inferenz weniger stark als Gewissheit und inbesondere
    nicht-monoton.
  \item Denn: Zwischen Ursachen und Effekten besteht keine $1$:$1$
    Relation (sondern $n$:$m$).
  \item D.h. Verschiedene Ursachen können denselben Effekt haben, und
    eine Ursache verschiedene Effekte.
  \end{itemize}
\end{frame}

\begin{frame}[plain]
  \begin{center}
    \includegraphics[height=\paperheight]{bicycledistorted.jpeg}
  \end{center}
\end{frame}

\begin{frame}
  \begin{itemize}[<+->]
  \item Verschiedene Objekte, Zustände könnten diesen Sinneseindruck
    verursachen:
    \begin{itemize}[<+->]
    \item Ein Fahrrad, das im Gebüsch liegt
    \item Einzelne Fahrradteile, die irgendwie im Gebüsch hängen
      geblieben sind
    \item Ein ungewöhnlich genau koordinierter Schwarm Bienen
    \end{itemize}
  \item Wie kommen wir vom Sinneseindruck zum (offensichtlichen)
    Ergebnis, dass hier ein Fahrrad im Gebüsch liegt?
  \end{itemize}

\end{frame}


\begin{frame}
  \begin{itemize}[<+->]
  \item Also: Wir brauchen eine \emph{Inferenz} auf die `beste' Ursache.
  \item Not any old inference will do: Besondere zusätzliche
    Beschränkungen auf die Art der Inferenz.
    \begin{itemize}[<+->]
    \item Hintergrundwissen des Agenten.
    \item Möglichkeit, gute/richtige Inferenz von schlechter/falscher
      abzugrenzen.
    \end{itemize}
  \end{itemize}
\end{frame}

\section{Bayesianistische Inferenz}

\subsection{Zutaten}

\begin{frame}
  \begin{itemize}[<+->]
  \item Gegeben Sinneseindruck und Hintergrundwissen soll auf die
    `richtige' Ursache des Sinneseindrucks geschlossen werden.
  \item Ranking der möglichen Ursachen nach Wahrscheinlichkeit.
  \item Die `beste' Ursache scheint guter Kandidat für den
    Wahrnehmungsinhalt zu sein.
  \end{itemize}

\end{frame}

\begin{frame} 
  Zutaten für die Bayesianistische Inferenz:

  \begin{description}[xxxxxxxxx]
  \item[$h_1,...,h_n$] Möglichen Hypothesen als Ursache eines
    Sinneseindrucks
  \item[$e$] Gegebener Sinneseindruck
  \item[$P(h_i)$] \emph{Prior:} Wahrscheinlichkeit einer Hypothese
    $h_i$, unabhängig davon, ob es die Ursache ist. Abhängig vom
    Hintergrundwissen.
  \item[$P(e|h_i)$] \emph{Likelihood:} Wahrscheinlichkeit, dass die in
    der Hypothese beschrieben Ursache so einen Sinneseindruck
    hervorrufen würde. Abhängig vom Hintergrundwissen. Maß dafür, wie
    gut eine Hypothese den Sinneseindruck \emph{vorhersagt}.
  \end{description}

  \begin{itemize}[<+->]
  \item Daraus lässt sich für jedes $h_i$ die bedingte
    Wahrscheinlichkeit gegeben $e$ errechnen.
  \end{itemize}

\end{frame}

\begin{frame}

  Vereinfaches Bayes Theorem \\{\scriptsize (Standardvariante folgt
    direkt aus der Definition von bedingter Wahrscheinlichkeit)}

  \begin{align*}
    P(h_i|e) &= P(e|h_i)P(h_i) \\
    \mathrm{Posterior \: Probability} &= \mathrm{Likelihood} \times  \mathrm{Prior \: Probability}
  \end{align*}

  \begin{itemize}[<+->]
  \item Hypothese $h_i$, für die $P(h_i|e)$ maximal ist, stellt
    plausibelste Ursache des Sinneseindrucks $e$ dar, gegeben das
    Hintergrundwissen des Agenten.
  \end{itemize}

\end{frame}

\subsection{Beispiel}

\begin{frame}
  Beispiel.
  \begin{description}[xxxxxxx]
  \item[$e$] {\small Sinneseindruck: Ein seltsames Klopfen.}
  \item[$h_1$] {\small Ein Specht klopft an der Wand.}
  \item[$h_2$] {\small Ein Einbrecher werkelt an der Tür.} \\ $...$
  \item[$h_{978}$] {\small Ich bin ein BIV und bekomme gerade einen
      elektrischen Stimulus entsprechend dem Stimulus.}
  \end{description}

  {\small Angenommen, ich habe zuletzt viele Spechte in der
    Nachbarschaft bemerkt, aber eher wenige Einbrecher. Ich habe mal
    ein Seminar zum Skeptizismus besucht und halte das ganze für großen Humbug. Dann gilt für die \emph{Prior Probabilities}}

  \begin{center}
  $P(h_1) > P(h_2) \gg P(h_{978})$.
  \end{center}

\end{frame}


\begin{frame}
  Was hat das mit Prediction (Vorhersage) zu tun?
\end{frame}

Zu intellektualistisch? -> Immediate Intuition von formaler
Erkenntnistheorie, in der Degress of Belief Bayesianistisch Updatet
werden. Hier aber perception: Inhalte mussen keinen propositionalen
Gehalt haben, nicht bewusst zugaenglich sein, usw.

\end{document}


