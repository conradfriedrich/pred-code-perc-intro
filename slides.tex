
\documentclass[11pt, handout]{beamer}
\usepackage[utf8]{inputenc}
\usetheme{Copenhagen} 
\usecolortheme{dove}
\usepackage{graphicx}
% \setbeamercovered{transparent}


\newcommand{\prule}{\vspace{-8pt}\rule{50pt}{0.5pt}}




\title[The Predictive Mind]{Jakob Hohwy: The Predictive Mind}
\subtitle{Chapters 1\&2}
\author{Conrad Friedrich}
\institute[Uni Köln]{Universität zu Köln}


\begin{document}

\begin{frame}[plain]
  \maketitle
\end{frame}

\section{Das Problem der Wahrnehmung}

\begin{frame}
  \begin{itemize}[<+->]
  \item Vereinheitlichende Theorie des Geistes: Wahrnehmung, Handlung,
    und ``alles mentale dazwischen'' (Auch Bewusstsein?).
  \item Zentrale Idee: Das Gehirn lässt sich als
    Hypothesen-Prüf-Mechanismus betrachten, der durchgehend damit
    beschäftigt ist, die Abweichung seiner Vorhersagen/Erwartungen
    (predictions) von seinen Sinneseindrücken zu \emph{minimieren}.
  \item Hier: Fokus auf Wahrnehmung.
  \item Die Sinneseindrücke formen Wahrnehmung nicht direkt, sondern
    sind Feedback zu den Erwartungen und Anfragen des Geistes ``an die
    Welt''.
  \end{itemize}
\end{frame}

\begin{frame}
  \begin{itemize}[<+->]
  \item Wahrnehmung besteht in (lässt sich am besten beschreiben als)
    unbewusster Inferenz auf die wahrscheinlichste \emph{Ursache}
    meiner \emph{rohen Sinneseindrücke}.
  \item Direkter, unmittelbarer Zugang nur zu den Sinneseindrücken,
    nicht zu den Dingen ``in der Welt''.
  \item \emph{Gewusst} (in einem starken Sinn, Gewissheit) werden nur
    die Effekte, d.h. Sinneseindrücke. Um etwas über die
    ``versteckten'' Ursachen zu erfahren, ist Inferenz nötig.
  \item Inferenz weniger stark als Gewissheit und inbesondere
    nicht-monoton.
  \item Denn: Zwischen Ursachen und Effekten besteht keine $1$:$1$
    Relation (sondern $n$:$m$).
  \item D.h. Verschiedene Ursachen können denselben Effekt haben, und
    eine Ursache verschiedene Effekte.
  \end{itemize}
\end{frame}

\begin{frame}[plain]
  \begin{center}
    \includegraphics[height=\paperheight]{bicycledistorted.jpeg}
  \end{center}
\end{frame}

\begin{frame}
  \begin{itemize}[<+->]
  \item Verschiedene Objekte, Zustände könnten diesen Sinneseindruck
    verursachen:
    \begin{itemize}[<+->]
    \item Ein Fahrrad, das im Gebüsch liegt
    \item Einzelne Fahrradteile, die irgendwie im Gebüsch hängen
      geblieben sind
    \item Ein ungewöhnlich genau koordinierter Schwarm Bienen
    \end{itemize}
  \item Wie kommen wir vom Sinneseindruck zum (offensichtlichen)
    Ergebnis, dass hier ein Fahrrad im Gebüsch liegt?
  \end{itemize}

\end{frame}

\section{Bayesianistische Wahrnehmung}

\begin{frame}
  \begin{itemize}[<+->]
  \item Also: Wir brauchen eine \emph{Inferenz} auf die (explanatorisch) beste Ursache.
  \item Not any old inference will do.
  \item 
  \end{itemize}
\end{frame}



\begin{frame}
Was hat das mit Prediction (Vorhersage) zu tun?

\end{frame}
\end{document}


